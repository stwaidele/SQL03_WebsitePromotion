% Vorgaben Assignment aus Studienheft SQL03
% Vorlage erstellt von Benutzer "Olfried" bei Fernstudenten.de
% Formatvorgaben fuer den Text
% Umfang: 8 - 10 Seiten (inkl. Abbildungen und Tabellen, aber ohne Deckblatt, % Gliederung und Literaturverzeichnis, Eidesstattliche Erklaerung)
% Zeilenabstand: 1,5
% Schriftart: frei
% Schriftgrad: 12 pt
% Variablen, physikalische Groessen und Funktionszeichen werden kursiv gedruckt.
% Korrekturrand: links: 4,5 cm, rechts 2,0 cm, oben und unten jeweils 3,0 cm
% Deckblatt: (Adresse, AKAD-E-Mail-Adresse, Immatrikulationsnummer, Modul-
% bezeichnung, Thema, Datum, Felder fÌr Korrektor)
% Gliederung (1 Seite)
% Literaturverzeichnis (3 - 5 Literaturquellen  z. B. Lehrbuecher, aktuelle Fachartikel recherchieren)
% Eidesstattliche Erklaerung (unterschrieben und fest eingebunden)
% Bearbeitungsdauer: 2 Monate


\documentclass[a4paper,12pt]{article}
\usepackage{german}
\usepackage[nottoc]{tocbibind} % Anzeigen des Literaturverzeichnisses im TOC
\usepackage{epsfig}
\usepackage{times}
\usepackage{supertabular}
\usepackage{wrapfig}
\usepackage{multirow}
\usepackage[onehalfspacing]{setspace}
\usepackage{listings}
\usepackage{mathptmx}
\usepackage{geometry}
\usepackage{helvet}
\usepackage{courier}
\usepackage{setspace}
\usepackage{textcomp}
\usepackage[T1]{fontenc}
\usepackage[latin9]{inputenc}
\usepackage{fancyhdr}
\usepackage{float} % Notwendig fuer figure[h]

\renewcommand{\familydefault}{\rmdefault}

\makeatother

\geometry{a4paper, left=45mm, right=20mm, top=30mm, bottom=30mm}

\begin{document}
\begin{center}

\vspace{9cm}

\Huge{Entwicklung von \\ Ma�nahmen f�r die Website--Promotion \\ am Beispiel der Kur+Reha GmbH}
\vspace{1cm}
\onehalfspacing

\Large{Vorbereitung auf das Seminar SQL03 \\  Wissenschaftliches Arbeiten}

\vspace{1cm}
%\Large{Betreuer: Prof. Dr. Paul Kirchberg \\}
\normalsize
\vspace{2cm}

Stefan Waidele\\Ensisheimer Stra�e 2\\79395 Neuenburg am Rhein
\\
\vspace{2cm}

11. Oktober 2012

Immatrikulationsnummer: 1028171

Stefan.Waidele@AKAD.de
\vspace{3cm}

%\epsfig{file=../../akad_logo.eps}


\end{center}



\pagestyle{empty} 

\onehalfspacing

\thispagestyle{empty}
\clearpage
\begin{center}
{\Large Eidesstattliche Erkl"arung}
\vspace*{4cm}\end{center}
\noindent
Ich versichere, dass ich das beiliegende Assignment selbstst"andig verfasst, keine anderen als die angegebenen Quellen und Hilfsmittel benutzt sowie alle w"ortlich oder sinngem"a"s "ubernommenen Stellen in der Arbeit gekennzeichnet habe. 
\vspace{3cm}

\hspace{-0.8cm}
\rule[0.5ex]{6.5cm}{1pt}
\hspace{1.3cm}
\rule[0.5ex]{6.5cm}{1pt}
(Datum, Ort)
\hspace{6.3cm}(Unterschrift)


\clearpage

\normalsize
\tableofcontents

\clearpage


\pagestyle{fancy}
\fancyhead{}
\fancyhead[LO,RE]{\textsc{Ma�nahmen zur Website Promotion}}
\fancyhead[RO,LE]{\thepage}
\fancyfoot[CO,CE]{}

\nocite{*} 

\setcounter{page}{1}



\section{Einleitung}
\subsection{Problemstellung und Ziel dieser Arbeit}
\subsection{Bestandsaufnahme: KURundREHA.de}
\subsection{Aufbau der Arbeit}

\section{Grundlagen}
\subsection{Definitionen und Abgrenzung}
\subsection{Methoden}

\section{Ziele des Website--Marketings}
\subsection{Allgemeine Ziele des Website--Marketings}
\subsection{Spezifische Ziele der Kur+Reha GmbH}
\subsection{Methoden der Erfolgskontrolle}

\section{Hauptteil: Instrumente des Website--Marketing}
\subsection{Instrument I}
\subsubsection{Allgemeine Betrachtung}
Ein Instrument mit Quellenangabe\footnote{\cite{Koj}}
\subsubsection{Relevanz f�r die Kur+Reha GmbH}
\subsubsection{Abgeleitete Ma�nahmen}
\subsection{Instrument II}
\subsubsection{Allgemeine Betrachtung}
Noch ein Instrument, ebenfalls mit Quellenangabe\footnote{\cite{Missfeld}}
\subsubsection{Relevanz f�r die Kur+Reha GmbH}
\subsubsection{Abgeleitete Ma�nahmen}
\subsection{Instrument III}
\subsubsection{Allgemeine Betrachtung}
Ein weiteres Instrument, ordentlich Beschrieben\footnote{\cite{Amersdorffer}} 
\subsubsection{Relevanz f�r die Kur+Reha GmbH}
\subsubsection{Abgeleitete Ma�nahmen}
\subsection{Weitere Instrumente}

\section{Schluss}
\subsection{Zusammenfassung}
\subsection{Kritische Auseinandersetzung}
\subsection{Betrachtung der Erfolgsfaktoren}
\subsection{Ausblick}

\bibliographystyle{apalike}
%\bibliographystyle{alpha}
\bibliography{sql03_websitepromotion}
\end{document}

